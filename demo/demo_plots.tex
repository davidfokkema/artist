\documentclass[a4paper,11pt]{article}

\usepackage{a4wide}
\usepackage[T1]{fontenc}
\usepackage{lmodern}

\usepackage{tikz}
\usetikzlibrary{arrows,pgfplots.groupplots,external}
\usepackage{pgfplots}
\pgfplotsset{compat=1.3}
\usepgfplotslibrary{polar}
\usepackage{graphicx}
\usepackage{relsize}

\usepackage[detect-family]{siunitx}
\usepackage[eulergreek]{sansmath}
\sisetup{text-sf=\sansmath}

\ifdefined\extrafloats
    \extrafloats{100}
\else
    \usepackage{morefloats}
\fi

\begin{document}

\begin{figure}
\centering
\includegraphics{stopping-power-mpl}
\caption{Stopping power of electrons and muons in a plastic scintillator.
Matplotlib-style.}
\end{figure}

\begin{figure}
\centering
\input{stopping-power}
\caption{Stopping power of electrons and muons in a plastic scintillator.}
\end{figure}

\begin{figure}
\centering
{\pgfkeys{/artist/width/.initial=.5\linewidth}
\input{shower-front}}
\caption{The measured arrival time distributions of vertical showers. The
showers are generated by a 1PeV proton.}
\end{figure}

\begin{figure}
\centering
\input{eas-lateral}
\caption{Lateral distribution of particles at sea level of an EAS
initiated by a \SI{1}{\peta\electronvolt} proton.}
\end{figure}

\begin{figure}
\centering
\input{histogram-fit}
\caption{Histogram of a sample of \num{2000} random numbers from a normal
distribution ($\mu = 0$, $\sigma = 1$).  The bin width is \num{.1}.  The
histogram is fitted with the pdf of the normal distribution, with
parameters $\mu = \num{3.9e-2}$, $\sigma=\num{1.0}$.}
\end{figure}

\begin{figure}
\centering
\input{spectrum}
\caption{Differential flux of primary cosmic rays as a function of
particle energy.  Figure is redrawn from Cronin, Swordy and Gaisser (Sci.
Am. 1, 1997).}
\end{figure}

\begin{figure}
\centering
{\pgfkeys{/artist/width/.initial=.7\linewidth}
\pgfkeys{/artist/height/.initial=.7\linewidth}
\input{sciencepark}}
\caption{Locations of stations and detectors in the Science Park Array.}
\end{figure}

\begin{figure}
\centering
\input{utrecht}
\caption{Locations of stations in the Utrecht HiSPARC cluster.
         Using map tile images as background. Map tiles by CartoDB,
         under CC BY 3.0. Data by OpenStreetMap, under ODbL}
\end{figure}

\begin{figure}
\centering
\input{multiplot}
\caption{Example of a multiplot.}
\end{figure}

\begin{figure}
\centering
\input{polar_histogram}
\caption{Example of histograms in a polar plot.}
\end{figure}

\begin{figure}
\centering
\input{discrete_directions}
\caption{All possible direction reconstructions with detectors in a
\SI{10}{\meter} equilateral triangle and \SI{2.5}{\nano\second} time
sampling. Example of radian units in polar plot.}
\end{figure}

\begin{figure}
\centering
\input{histogram2d}
\caption{2D histograms. Top: vectorized, bottom: bitmap, left:
reverse\_bw, and right: bw.}
\end{figure}

\begin{figure}
\centering
\input{multi_histogram2d}
\caption{2D histograms (MultiPlot version). Top: vectorized, bottom:
bitmap, left: reverse\_bw, and right: bw.}
\end{figure}

\begin{figure}
\centering
\input{event_display}
\caption{A detected air shower by station 503. The size of the points
indicates the number of particles in each detector. The color is based
on the arrival time of the particles.}
\end{figure}

\begin{figure}
\centering
\input{multi_event_display}
\caption{Two detected air showers, one by station 503 and one by station
508. The size of the points indicates the number of particles in each
detector. The color is based on the arrival time of the particles.}
\end{figure}

\begin{figure}
\centering
\input{relative_pin}
\caption{This shows how the relative position for pins can be used.
         The value (between 0 and 1) is a fraction of the path length
         described by the x, y values. So .5 is at 50\% of the path length.}
\end{figure}

\begin{figure}
\centering
\input{relative_pin_log}
\caption{This shows how the relative positions are calculated for
         logarithmic plots. The 'seen' path length is used to determine
         the relative position. When using the \texttt{x=<value>} option, the
         position is correctly interpolated.}
\end{figure}

\begin{figure}
\centering
\input{any_option}
\caption{This shows that any other pgfplots option like legends and
         grids can be enabled manually.}
\end{figure}

\begin{figure}
\centering
\input{multi_any_option}
\caption{This shows that pgfplots axis option like grids and automatic
         limit enlargement can be set manually for individual subplots
         and globally for all subplots.}
\end{figure}

\begin{figure}
\centering
\input{error_bars}
\caption{This shows an example of symmetric and asymmetric error bars.}
\end{figure}

\begin{figure}
\centering
\input{fourier_with_legend}
\caption{You can add a legend to a plot by using the legend parameter.}
\end{figure}

\begin{figure}
\centering
\input{fourier_with_labels}
\caption{Or, if you prefer, you can use labels.}
\end{figure}

\begin{figure}
\centering
\input{mm_paper}
\caption{You can include graph paper as background for your plot. It is important that you correctly specify axis limits and scale to line up everything. This is, in fact, very intuitive. This is especially useful in high school physics tests and exams. This graph was included in an exam question on magnetic resonance imaging (MRI). An EM-pulse containing many frequencies is transmitted through a patient. Due to a gradient field, hydrogen nuclei resonance frequencies depend on the position inside the MRI-scanner. The decay signal from the nuclei show a peak at \SI{70}{\mega\hertz} which can be translated into a position inside the patient's body.}
\end{figure}

\begin{figure}
\centering
\input{mm_paper_multiplot}
\caption{Graph paper is also supported for MultiPlots.}
\end{figure}

\end{document}
